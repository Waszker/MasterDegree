\documentclass[12pt,twoside,a4paper]{article}
% ten dokument należy kompilować za pomocą XeLaTeX-a. Trzeba mieć zainstalowane w systemie czcionki Arial i Arial Narrow

\usepackage[utf8]{inputenc}
\usepackage[small,sf,bf]{titlesec}

\usepackage{indentfirst}
\setlength{\parindent}{5mm}

\usepackage{hyphenat}
\hyphenation{in-for-ma-cyj-nych}

\usepackage{polski}
\usepackage[T1]{fontenc}
\usepackage{graphicx}
\usepackage{geometry}
\usepackage{a4wide}
\geometry{left=20mm,right=20mm,bindingoffset=10mm, top=25mm, bottom=25mm}




\linespread{1.5}
 
 
\usepackage{amsfonts}
\usepackage{amsmath,amssymb,amsthm}
\usepackage{latexsym,xpatch}
\usepackage{mathrsfs}
\usepackage{enumerate}
\usepackage{verbatim}
\usepackage{textcomp}
\usepackage{multirow}



\def\university{POLITECHNIKA WARSZAWSKA}
\def\faculty{WYDZIAL MATEMATYKI I NAUK \\ INFORMACYJNYCH}
\def\type{magisters}
\def\discipline{Informatyka}
\def\spec{Metody Sztucznej Inteligencji}
\def\title{Rejection Option in Pattern Recognition Problem - Selected Issues}
\def\author{inż. Piotr Waszkiewicz}
\def\supervisor{dr hab. inż. Władysław Homenda}
\def\album{254218}
\def\year{2017}

\begin{document}
\sloppy
\pagestyle{empty}



\includegraphics[scale=1.]{Figures/politechnika} 

\begin{center}
\vspace{70pt}


\includegraphics[scale=1.]{Figures/praca_mgr} % 

{  na kierunku \discipline

\vspace{40pt}
{ \large \title}

\vspace{50pt}

{ \huge \author}

\vspace{5pt}

Numer albumu \album

\vspace{40pt}

promotor \\
{ \supervisor}

\vspace{15pt}
 

 \vfill
WARSZAWA \year \\
}
\end{center}


\newpage
\null

\vfill

\begin{center}
\begin{tabular}[t]{ccc}
............................................. & \hspace*{100pt} & .............................................\\
podpis promotora & \hspace*{100pt} & podpis autora
\end{tabular}
\end{center}


\end{document}